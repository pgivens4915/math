% Paul Givens

\documentclass[letterpaper]{article}

\usepackage{ amssymb }
\usepackage{mathrsfs}

\title{Exercises from Mordechai}
\author{Paul Givens}

\setlength\parindent{0pt}

\begin{document}
\maketitle
2.9) 

\textbf{Show:}
$ (p \wedge q \rightarrow r) \vDash (p \rightarrow r) \vee (q \rightarrow r)) $
\\[1ex]

$ \neg (p \wedge q) \vee r \vDash \neg p \vee (\neg q \vee r) $

$ \neg p \vee \neg q \vee r \vDash \neg p \vee \neg q \vee r $ 

\textbf{Q.E.D.}
\\[1ex]

\textbf{Show:}
$ p \wedge q \rightarrow r \nvDash p \rightarrow r $
\\[1ex]

If we are to show that $ p \rightarrow r $ is not a logical consequence of
$ p \wedge q \rightarrow r $ we must show that $ \exists $ a model
$ \mathscr{I} $ that satisfies $ p \wedge q \rightarrow r $ and does not 
satisfy $ p \rightarrow r $ this is because by definition 
$ U \vDash p $ iff every model U is a model of p 

let $ \mathscr{I} = \{ p = t, q = f, r = f\}$ 

$ \mathscr{I} $ is true for $ p \rightarrow r $ and false for 
$ p \wedge q \rightarrow r $

\textbf{Q.E.D.}

\end{document}
